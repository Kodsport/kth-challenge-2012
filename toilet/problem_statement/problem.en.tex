\problemname{Toilet Seat}
Many potential conflicts lurk in the workplace and one of the
most sensitive issues involves toilet seats. Should you leave the
seat ``up'' or ``down''? This also affects productivity, particularly at
large companies. Hours each week are lost when employees need
to adjust toilet seats. Your task is to analyze the impact different
bathroom policies will have on the number of seat adjustments
required.

The classical assumption is that a male usually uses a toilet
with the seat ``up'' whereas a female usually uses it with the seat
``down''. However, we will divide the population into those who
prefer the seat up and those who prefer it down, regardless of gender.

Now, there are several possible policies that one could use, here are a few:
\begin{enumerate}
    \item When you leave, always leave the seat up
    \item When you leave, always leave the seat down
    \item When you leave, always leave the seat as you would like to find it
\end{enumerate}
So, a person may have to adjust the seat prior to using the toilet and, depending on policy, may need to adjust
it before leaving.

\section*{Task}
Your task is to evaluate these different policies. For a given sequence of people's preferences, you are supposed to
calculate how many seat adjustments are made for each policy

\section*{Input}
The first and only line of input contains a string of characters '\texttt{U}' and '\texttt{D}', indicating that a person in the sequence wants the seat up or down.
The string has length at least $2$ and at most $1000$.

The first character indicates the initial position of the toilet seat, and the following $n - 1$ characters indicate how a sequence of $n - 1$ people prefer the seat.
You should compute the total number of seat adjustments needed for each of the three policies described above.

\section*{Output}
Output three numbers, each on a separate line, the total number of seat adjustments for each policy.
