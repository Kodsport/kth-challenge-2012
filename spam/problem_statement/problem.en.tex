\problemname{Spam Filter}
\illustration{0.3}{spam.jpg}{Photo by \href{https://www.flickr.com/photos/jeepersmedia/14014574484/in/photolist-nmqmJy-ym8dnV-ym27yU-xFKTsV-ym27j5-no9xSV-7fTCpn-yCD2gB-nmqnEG-98zP97-6XmzT8-no9x2X-3ei882-nm6EZX-no9wVe-bQUPqg-5BLW6G-69RAFh-77g1rz-teYAE-9i4Ken-5UGH4w-25fM1F-5W2YE1-7WCyEc-7WCyG4-fKRqS-8AeGgj-5kFD98-25fM1r-6HD6q5-bq1gWk-5edh4w-5V6dce-4tkzcR-4wryaZ-8HRzKd-5BbPkc-7NdeUG-8zo9EP-8Ywrzb-5BbPza-25fLYX-2ihU9t-25fM1a-4cGdeu-6xuFAN-tuDnR-7N9eXr-7BuiiD}{Mike Mozart}, \href{https://creativecommons.org/licenses/by/2.0/}{CC BY 2.0}}

\noindent
Goo is working in a well-known Slovak antivirus company
which unfortunately cannot be named. In addition to antivirus
software, they are developing a spam filter. Recently, Goo has
made a few improvements to the filter and he wants to demonstrate
his progress to his boss. As you can imagine, demonstrating lowlevel ideas of your implementation is not a good way to impress
your boss, so Goo instead decided to make a presentation with
plenty of graphs showing filtering results. The company has a
huge database of e-mails and each e-mail is marked as a spam or
ham (i.e. not spam). These e-mails were all correctly marked by
people -- every time someone in the company receives an e-mail,
he marks it as either spam or ham and adds it to the database.

The success of Goo's program can be measured in a simple
way. Goo ran his program on all e-mails in the database. For each
message he noted if his program correctly decided whether the message was spam or ham. The messages were
processed in order from the oldest to the newest one. To impress the boss, Goo wants to select e-mails from a
period of time and calculate the success rate only for this period. Of course, a period containing only one e-mail
won’t impress anyone, so Goo wants to choose a period which is long enough.


\section*{Task}
You are given a sequence of test results and a number $k$. Your task is to find a continuous subsequence of length at
least $k$ which has the highest possible success rate among all such subsequences. The success rate of a subsequence
is defined as the number of successfully classified e-mails divided by the length of the subsequence.

\section*{Input}
On the first line there is an integer $k$ ($1 \le k \le 100$) denoting the minimal subsequence length. The second line
contains a string consisting of characters \texttt{0} and \texttt{1}, denoting answers of the program for each e-mail in the database.
Number \texttt{1} indicates that Goo’s program gave a correct answer and \texttt{0} that it failed. The length of the string will be
at least $k$ and at most $100\,000$ characters.

\section*{Output}
The first and only line of output should consist of two integers $f$ and $l$, separated by a single space.
The integer $f$ is the $1$-based index of the first element of subsequence with the best success rate and $l$ is its length.
If there are multiple optimal solutions, you can output any one of them.
